\section{Описание}
Требуется написать реализацию $AVL-$дерева. Идея его заключается в том, что у каждой вершины разница высот между левым и правым деревом по модулю не превосходит единицы. Также это дерево является бинарным деревом поиска. При вставке и удалении элементов всегда нужно проверять баланс вершин, у которых он мог измениться, и при необходимости делать балансировку с помощью поворотов. Все действия очень похожи на действия с бинарным деревом.

\pagebreak

\section{Исходный код}

Все функции схожи с функциями для бинарного дерева поиска. Отличием является балансировка, которая требует дополнительного написания функций и внесения новых шагов в функции. Также в структуру ноды я добавил поле для баланса и ссылку на предка, чтобы было легче совершать балансировку. Алгоритм поворотов взят прямиком из  книги Кормена: в зависимости от типа поворота (правый или левый) левый или правый ребёнок текущей ноды становится на её место, а нода занимает место ребёнка (функции $LeftRotate$ и $RightRotate$). Повороты используются в функции $BalancingForInsert$, когда в ноде баланс равен 2 или -2. После произведённых поворотов баланс становится нулём и дерево снова может считаться $AVL-$деревом. В добавлении (функция $Insert$) я производил изменение баланса только после того, как вставил элемент, поэтому взаимодействие с балансами идёт через родителей. Условие остановки ребалансировки для добавления: баланс одной из нод равен нулю, либо дошли до корня. В это случае возвращаем $true$ и больше балансы не трогаем. Примерно тот же алгоритм и в удалении. Только здесь я уже по ходу поиска ноды я меняю балансы, и при необходимости возвращаю их назад (если ниже баланс уже скорректирован). Такой подход связан с множествами особенностей удаления. В $BalancingForDelete$ мы рассматриваем больше случаев, чем в $BalancingForInsert$(опять же, из-за особенностей удаления). В функции $GoingDeeper$ идёт поиск замены текущей ноде (самый левый элемент правого дерева). Там также происходит ребалансировка, и если выполнены условия конца ребалансировки (для удаления это: стоим в корне, либо баланс текущей ноды равен 1 или -1) возвращаем $true$ и больше балансы не трогаем. В функции $Save$ открываем файл с указанным названием на вывод и с помощью очереди печатаем всё дерево в файл (ключи и значение). Псевдо-бфс. В функции $Load$ мы открываем файл с указанным названием на поток ввода, считываем данные и добавляем их в дерево. Перед считываем и добавлением чистим всё дерево.

\begin{longtable}{|p{7.5cm}|p{7.5cm}|}
\hline
\rowcolor{lightgray}
\multicolumn{2}{|c|} {lab2.cpp}\\
\hline
void DestroyTAVLTreeNode(TAVLTree
Node *x)& Удаление ноды дерева\\
\hline
void DestroyTAVLTree(TAVLTreeNode *x)& Удаление всего дерева\\
\hline
TAVLTree::$\sim$TAVLTree() & Деструктор дерева\\
\hline
void TAVLTree::LeftRotate(TAVLTree
Node *x)& Левый поворот\\
\hline
void TAVLTree::RightRotate(TAVLTree
Node *y)& Правый поворот\\
\hline
void TAVLTree::Insert(string key, 
unsigned long long value, 
bool \&status)& Добавление в дерево\\
\hline
bool TAVLTree::BalancingFor
Insert(TAVLTreeNode *node)& Балансировка для добавления\\
\hline
bool TAVLTree::Insert( string key, 
unsigned long long value, 
TAVLTreeNode *node, bool \&status )& Добавление в дерево элемента, если
корень уже есть\\
\hline
void PrintAVLNodeElems(TAVLTreeNode *x, 
int h)& Печать всех элементов дерева\\
\hline
void TAVLTree::Print()& Печать дерева\\
\hline
bool TAVLTree::Search(string key, 
TAVLTreeNode * node)& Поиск ключа в дереве\\
\hline
void TAVLTree::Search(string key)& Поиск\\
\hline
bool TAVLTree::BalancingForDelete
(TAVLTreeNode* node)& Балансировка для удаления\\
\hline
bool TAVLTree::GoingDeeper(TAVLTree
Node *node, string\& key, 
unsigned long long\& value)& Поиск замены удаляемой ноде\\
\hline
void TAVLTree::Delete(string key)& Удаление ноды\\
\hline
bool TAVLTree::Delete( string key, 
TAVLTreeNode *node )& Удаление ноды, если она не корень\\
\hline
void TAVLTree::Save(string filename)& Запись дерева в файл\\
\hline
void TAVLTree::Load(string filename)& Загрузка дерева из файла\\
\hline
\rowcolor{lightgray}
\hline
\end{longtable}

\begin{lstlisting}[language=C]
struct TAVLTreeNode {
    string key;
    unsigned long long value;
    short balance;
    TAVLTreeNode *left;
    TAVLTreeNode *right;
    TAVLTreeNode *parent;
    TAVLTreeNode(string key1, unsigned long long value1, TAVLTreeNode *left1, TAVLTreeNode *right1, TAVLTreeNode *parent1, short balance1) {
        key = key1;
        value = value1;
        left = left1;
        right = right1;
        parent = parent1;
        balance = balance1;
    }
};

class TAVLTree {
public:
    TAVLTree() {}
    ~TAVLTree();
    void Insert(string key, unsigned long long value, bool &status);
    void Search(string key);
    void Delete(string key);
    void Print();
    void Save(string filename);
    void Load(string filename);
private:
    bool Insert(string key, unsigned long long value, TAVLTreeNode * node, bool &status);
    bool Search(string key, TAVLTreeNode * node );
    void LeftRotate(TAVLTreeNode *node);
    void RightRotate(TAVLTreeNode *node);
    bool BalancingForInsert(TAVLTreeNode *node);
    bool BalancingForDelete(TAVLTreeNode *node);
    bool GoingDeeper(TAVLTreeNode* node, string& key, unsigned long long& value);
    bool Delete(string key, TAVLTreeNode * node);
private:
    TAVLTreeNode *root = nullptr;
};
\end{lstlisting}


\section{Консоль}
\begin{alltt}
root@DESKTOP-5HM2HTK:~# cat <test
+ a 1
+ A 2
+ aaaaaaaaaaaaaaaaaaaaaaaaaaaaaaaaaaaaaaaaaaaaaaaaaaaaaaaaaaaaaaaaaaaaaaaaaaaaaaaaaaaaaaaaaaaaaaaaaaaaaaaaaaaaaaaaaaaaaaaaaaaaaaaaaaaaaaaaaaaaaaaaaaaaaaaaaaaaaaaaaaaaaaaaaaaaaaaaaaaaaaaaaaaaaaaaaaaaaaaaaaaaaaaaaaaaaaaaaaaaaaaaaaaaaaaaaaaaaaaaaaaaaaaaaaaaaaaa 18446744073709551615
aaaaaaaaaaaaaaaaaaaaaaaaaaaaaaaaaaaaaaaaaaaaaaaaaaaaaaaaaaaaaaaaaaaaaaaaaaaaaaaaaaaaaaaaaaaaaaaaaaaaaaaaaaaaaaaaaaaaaaaaaaaaaaaaaaaaaaaaaaaaaaaaaaaaaaaaaaaaaaaaaaaaaaaaaaaaaaaaaaaaaaaaaaaaaaaaaaaaaaaaaaaaaaaaaaaaaaaaaaaaaaaaaaaaaaaaaaaaaaaaaaaaaaaaaaaaaaaa
A
- A
a
root@DESKTOP-5HM2HTK:~# g++ lab2.cpp
root@DESKTOP-5HM2HTK:~# ./a.out <test
OK
Exist
OK
OK: 18446744073709551615
OK: 1
OK
NoSuchWord
root@DESKTOP-5HM2HTK:~#
\end{alltt}
\pagebreak

